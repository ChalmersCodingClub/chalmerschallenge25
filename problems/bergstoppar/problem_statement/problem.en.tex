\problemname{Bergstoppar}
\noindent

Joel has spent a lot more time climbing mountains than programming the last 6 months, and he once again finds himself on top of another mountain.
This time he has climbed a very special mountain range, where all the mountain tops are on a single line. 
The mountain Joel has climbed is $h$ meters high, and he is at relative horizontal position 0 in the mountain range.
Given the coordinates of $n$ other mountains in the range, you need to calculate how many of them Joel can actually see.
You should assume that Joel is laying on the ground when looking for other mountains, hugging the mountain he has climbed (he does after all like the mountains very much).

\section*{Input}
The first line will contain $n$, the number of other mountains to consider. $(1 \leq n \leq 50\;000)$

The second line will contain $h$, the height of the mountain Joel is at. $(0 \leq h \leq 8848)$

The following $n$ lines will each contain the coordinates of another mountain top, in the format $x$ $y$, where $x$ is the horizontal relative position to Joel,
and $y$ is the height of the mountain. $(-50\;000 \leq x \leq 50\;000)$, $(0 \leq y \leq 8848)$.

All coordinates have integer values, Please note that $x$ may be negative!

\section*{Output}
The output should consist of a single integer, the number of other mountain tops Joel can see.

\section*{Points}
Your solution will be tested on several test case groups. To get the points for
a group, it must pass all the test cases in the group.

\noindent
\begin{tabular}{| l | l | p{12cm} |}
  \hline
  \textbf{Group} & \textbf{Point value} & \textbf{Constraints} \\ \hline
  $1$    & $10$       & $N = 2,\;x > 0$ \\ \hline
  $2$    & $30$       & $N \leq 1000,\;x > 0$ \\ \hline
  $3$    & $10$       & $N \leq 1000$ \\ \hline
  $4$    & $50$       & No Additional Constraints \\ \hline
\end{tabular}





\problemname{Building a Castle}
\noindent

Your sister wants to build a castle. She has $N$ blocks, each with a unique
height from $1$ to $N$. Since she just tore down her last castle, they are currently 
laid out in a uniformly random order.

Her new castle must have 4 towers. She will consider each block in the 
given order. For each block, she can either discard it entirely, or add
it to one of the towers. A block can be placed on top of a tower if the tower is empty or
if the height of the new block is at least $K$ greater than the height of the topmost
block in that tower. Once a block is discarded, it cannot be used
later.

Since her sense of architectural aesthetics are not yet developed, she only
cares about maxiziming the number of blocks in her towers. Find the maximum
number of blocks she can place in her towers.

\section*{Input}
The first line of input contains the integers $N$ and $K$ ($1 \leq K \leq N \leq 10^5$),
the number of blocks and the required height difference between blocks.

The following line will contain $N$ integers, the order of the blocks. It is guaranteed
that the order was generated in such a way that all $N!$ possible orders are equally likely.
Note that $N$ and $K$ are not selected randomly.

\section*{Output}
Print an integer: the largest possible total number of blocks in her four towers.


\section*{Scoring}
Your solution will be tested on a set of test groups, each worth a number of points. Each test group contains
a set of test cases. To get the points for a test group you need to solve all test cases in the test group.

\noindent
\begin{tabular}{| l | l | p{12cm} |}
  \hline
  \textbf{Group} & \textbf{Points} & \textbf{Constraints} \\ \hline
  $1$    & $6$        & $N \leq 30$ \\ \hline
  $2$    & $14$       & $N \leq 60$ \\ \hline
  $3$    & $15$       & $N \leq 300$ \\ \hline
  $4$    & $20$       & $N \leq 3000$ \\ \hline
  $5$    & $25$       & $N \leq 20000$ \\ \hline
  $6$    & $20$       & No additional constraints. \\ \hline
\end{tabular}

\section*{Explanation of Sample 1}
Since $K=1$, the only requirement is that blocks are in increasing height in each tower.
The block [5 1 2 6 3 7 8 4] can divided into towers [5 6 7 8], [1 2 3 4], [] and [].

\section*{Explanation of Sample 2}
Once again, $K=1$. It is impossible for any tower to have two blocks. Therefore, each one has 1 block and the answer is
4.

\section*{Explanation of Sample 3}
It is impossible to build a castle with all blocks. One possible optimal division of the blocks [7 8 10 5 2 1 9 4 3 6]
between the towers is [7 10], [5 9], [2 4 6] and [1 3]. Note that if $K$ were 1, it would be possible to build a castle
with all blocks.

\section*{Explanation of Sample 4}
Despite $K$ being large, we can build a castle with all blocks, since we do not care about $K$ for the bottommost block
in each tower.

Note that the samples were not randomly generated. Therefore, your solution does not need to solve the samples to get points
on the problem. It is guaranteed that the rest of the testcases was generated in such a way that every permutation of size
$N$ is equally likely.

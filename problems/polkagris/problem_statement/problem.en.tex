\problemname{Polkagrios}
\noindent

The board of Chalmers Coding Club have departed from Gothenburg on tågluff to Hamburg, Zürich, Venice and
beyond, but the first stop is Gränna! Gränna is famous for its polkagris, which is similar to candy
canes, but are available in many different colors and flavors. The colors of the sticks are typically
twisted together, but in order for this problem to make sense you have to imagine they are sequences
of single-colored segments.

On their way there, the board has spent a great deal of time deciding the flavors for their desired
polkagris, which can be represented as a string of $T$ letters $A-Z$. But upon arrival to the
polkagris shop, all their sticks are sold out except for a few belonging to one same batch, that,
unfortunately, does not match the desired configuration.

A little distraught, but not that much since the trip will be very fun regardless, the board leaves
to continue their journey without polkagrisar. That night, Gustav, who has in recent times become
impressively culinarly minded, realizes that maybe they could have reheated the polkagris while
folding one of its ends over and onto itself, kneading it into a single stick again. Maybe this
process, if performed iteratively, could have turned the store's polkagris into the one that they
actually wanted?

\section*{Input}

The first line of input contains the two integers $N$ and $T$ $(1 \leq t \leq n \leq 3 \cdot 10^7)$,
the lengths of the store's available and the board's desired polkagrisstång. 

No two adjacent characters will be identical.

\section*{Output}
If the available polkagrisstång can be worked into the desired one, print ``possible''. Otherwise, print ``impossible''.

\section*{Scoring}
Your solution will be tested on a set of test groups, each worth a number of points. Each test group contains
a set of test cases. To get the points for a test group you need to solve all test cases in the test group.

\noindent
\begin{tabular}{| l | l | p{12cm} |}
  \hline
  \textbf{Group} & \textbf{Points} & \textbf{Constraints} \\ \hline
  $1$    & $27$       & $N \leq 100$ \\ \hline
  $2$    & $31$       & $N \leq 10^4$ \\ \hline
  $3$    & $42$       & No additional constraints. \\ \hline
\end{tabular}

\section*{Explanation of sample 1}

The string \verb|ABCBC| can be folded at its midpoint to produce something that after kneading looks
like \verb|AABBC|, but the board does not mind unevenly long or thick segments and so considers this
an acceptable \verb|ABC| polkagris.

\section*{Explanation of sample 2}

The polkagris \verb|ABC| is of course the same as \verb|CBA|, just turn it the other way!

\section*{Explanation of sample 3}

The leftmost flavor of \verb|ABABABCA| can be repeatedly folded over to reach \verb|ABCA|, but since
there is no way to fold away the rightmost \verb|A| creating the polkagris \verb|ABC| is
unfortunately not possible.

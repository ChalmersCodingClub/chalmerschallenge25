\problemname{Passing Time}
Hugo has just finished his exam week and is really bored, as the new courses don't start until Monday.
To pass the time, Hugo has started playing a solitaire card game. The game works as follows:

Initially, Hugo has a standard shuffled deck of $52$ cards in front of him. In each round, Hugo guesses the rank of the top card (3, 10, jack, etc).
After he makes his guess, he flips over the top card, and if he guessed correctly, he scores a point, and the next round begins with the next card on top.

Help Hugo figure out how likely it is that he guesses correctly the $N+1$:th round (that is to say, after $N$ cards have been drawn),
given that he plays optimally.

\section*{Input}
The first line contains an integer $N$ ($0 \le N \le 51$), the number of rounds that have been played so far.

The following $N$ lines each describe which card has appeared in each round, e.g. \texttt{10D}, \texttt{4C}, \texttt{KS}.
The cards are described by a rank (\texttt{A}, \texttt{2}, \texttt{3}, \texttt{4}, \texttt{5}, \texttt{6}, \texttt{7}, \texttt{8},
\texttt{9}, \texttt{10}, \texttt{J}, \texttt{Q}, \texttt{K}) followed by a suit (\texttt{C}, \texttt{S}, \texttt{D}, \texttt{H}).
It is guaranteed that no card appears twice.

\section*{Output}
Print a decimal number between 0 and 1: the probability that Hugo guesses correctly.

Your answer will be considered correct if the absolute difference from the correct answer is less than $10^{-3}$.

\section*{Explanation of sample 2}
If Hugo guesses that the next card shown is going to be a $4$, the probability that
he is correct will be $4/47 \approx 0.0851$, since there are $4$ remaining $4$:s in the deck out of $47$ remaining cards.
It can be shown that there is no other guess that will exceed this probability of being correct.

\section*{Scoring}
Your solution will be tested on a set of test groups, each worth a number of points. Each test group contains
a set of test cases. To get the points for a test group you need to solve all test cases in the test group.

\noindent
\begin{tabular}{| l | l | p{12cm} |}
  \hline
  \textbf{Group} & \textbf{Points} & \textbf{Constraints} \\ \hline
  $1$    & $20$       & $N < 10$ \\ \hline
  $2$    & $80$       & No additional constraints. \\ \hline
\end{tabular}

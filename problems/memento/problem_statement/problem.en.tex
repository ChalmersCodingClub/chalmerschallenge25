\problemname{Memento}
\noindent
Unfortunately, Loke has a very bad memory. Sometimes, he may even completely forget that he's seen a given graph. 
To avoid this public embarassment, he's going to need a system. Every time he sees a graph, he can add at most $30$
edges to it. However, the next time he sees it, he needs to be able to identify that he's seen it before based on the
change. To make matters even worse, the graph is very large- it has $1000$ vertices and between $3500$ and $4500$ edges.
To make things easier for him, he has asked for your help in writing a computer program to help him.

More specifically, you're asked to implement a computer program that does the following:
\begin{enumerate}
  \item Reads an undirected graph from standard input.
  \item Determines whether the program has seen the graph before.
  \item If he hasn't seen it before, the program should print ``not seen before'', then print the $30$ edges to be added and exit.
  \item If he has it before, the program should print ``seen before'' and exit.
\end{enumerate}
To explain the process more exactly, the judge will:
\begin{enumerate}
  \item The judge will generate a graph (exactly how is described below).
  \item The judge will start your program and give it the graph. If it incorrectly claims that it's seen the graph before,
  you will receive a ``Wrong Answer'' verdict. Otherwise, the judge will then add the $30$ edges to the graph.
  \item The judge will then shuffle the vertex labels (i.e. node $1$ might be changed to $5$, but the new
  graph will be isomorphic to the one before the shuffle). The order in which edges are given, in addition the
  order between the vertices of the edge will also be shuffled (i.e. $4 2$ may be changed to $2 4$).
  \item The judge will then start your program and give it the shuffled graph. If it correctly identifies that
  it's seen the graph before, you have solved this testcase. Otherwise, you will receive a ``Wrong Answer'' verdict.
\end{enumerate}

The method for generating the initial graph is as follows:
\begin{verbatim}
G = empty graph with 1000 nodes
choose M = random number between 3500 and 4500
for i between 1...M:
  add a random edge to G that doesn't already exist and connects two distinct vertices
\end{verbatim}
A python implementation of this process may be found in attachments. Note that the real judge uses a different pseudo-random
number generator, but it is guaranteed that it is of comparable quality to Python's random number generator.


\section*{Input}
The first line of input contains the integer $M$ ($3500 \leq M \leq 4550$), the number of edges in the graph. Note that
the number of nodes is always $1000$.

The following $M$ lines each contain the integers $a$ and $b$ ($0 \leq a,b \leq 999$, $a \neq b$), meaning that there is an edge
between the nodes with index $a$ and $b$. Additionally, there will be at most one edge between any pair of vertices.

\section*{Output}
If you think you've seen the graph before, print only ``seen before''.

Otherwise, print ``not seen before''. On the following $30$ lines, print $a b$, meaning that you want to add
an edge between node $a$ and $b$. Note that $0 \leq a,b \leq 999$ and $a \neq b$ must hold. In addition, after
all edges have been added, there must not exist a pair of nodes with more than one edge between them.

\section*{Scoring}
Your solution will be tested on 25 testcases. If you pass all of them, your solution will be accepted.
The jury guarantees that there exists a solutions that pass all testcases with very high probability.

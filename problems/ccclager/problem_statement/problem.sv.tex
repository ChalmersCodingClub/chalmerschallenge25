\problemname{CCC läger}

Det är dags för ett nytt läger i Härryda med Chalmers Coding Club och som tradition har alla bastat till klockan 3 på
morgonen och har nu gått och lagt sig. Alla på lägret sover i ett stort avlångt rum (Storstugan) men det finns dessvärre
också ett fönster längs långsidan riktad rakt mot där morgonsolen snart dyker upp. Mer specifikt har lägret $N$ stycken
deltagare där den $i$:te deltagaren sover $d_i$ meter från dörren (på kortsidan) av rummet.

\illustration{0.5}{sample_bild.png}{Illustrering av exempel $1$, med gardiner markerade i rött}

Rasmus har vaknat tidigt och märker att det finns ett antal gardiner som täcker delar av fönstret på $M$ olika ställen,
gardin bit $j$ täcker solen för alla som sover mellan $L_j$ och $R_j$ meter från dörren, så att de kan sova vidare. Alla
som blir träffade av den starka morgonsolen kommer däremot direkt att vakna.

För att hjälpa alla stackars deltagare som kommer att vakna tidigt tänker Rasmus sätta på kaffe. Han tänker att varje
deltagare kommer att behöva en kopp kaffe för att vakna ordentligt, men han vet också att Joshua och Gustav
var vakna ännu längre för att förbereda morgondagens föreläsningar, så de kommer att behöva två koppar kaffe. Hur
många koppar kaffe måste Rasmus sätta på?


\section*{Indata}
Första raden består av två tal $N$ ($1 \le N \le 2 * 10^5$) och $M$ ($0 \le M \le 2 * 10^5$), antalet medlemmar och antalet bitar gardin. Detta följs
av $N$ rader, den $i$:te raden innehåller namnet på den $i$:te deltagaren följt av $d_i$ ($0 \le d_i \le 10^9$), hur långt
från dörren de sover. Varje namn består av 3 till 10 bokstäver av tecken mellan a-z, där alla bokstäver är små förutom första.
Sedan följer $M$ rader, där den $j$:te raden innehåller två tal, $L_j$ och $R_j$ ($0 \le L_j \le R_j \le 10^9$),
intervallet som den $j$:te gardinen täcker.

\section*{Utdata}
Skriv ut ett tal, antalet koppar kaffe som Rasmus måste sätta på för att tillfredställa de nyvakna deltagarna.

\section*{Poängsättning}
Din lösning kommer att testas på en mängd testfallsgrupper.
För att få poäng för en grupp så måste du klara alla testfall i gruppen.

\noindent
\begin{tabular}{| l | l | p{12cm} |}
  \hline
  \textbf{Grupp} & \textbf{Poäng} & \textbf{Gränser} \\ \hline
  $1$    & $10$       & $N, M \le 100$; $d_i, L_j, R_j \le 1000$ \\ \hline
  $2$    & $20$       & $M \le 100$; $d_i, L_j, R_j \le 10^5$ \\ \hline
  $3$    & $20$       & $d_i, L_j, R_j \le 10^5$ \\ \hline
  $4$    & $50$       & Inga ytterligare begränsningar. \\ \hline
\end{tabular}
